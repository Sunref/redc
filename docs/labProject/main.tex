\documentclass[12pt, a4paper]{article}

% --- BLOCO DE PREÂMBULO UNIVERSAL (Adaptado para o User) ---
\usepackage[a4paper, top=3cm, bottom=2cm, left=3cm, right=2cm]{geometry}
\usepackage{fontspec}

% Configuração de Idioma (Português do Brasil)
\usepackage[brazilian, provide=*]{babel}
\babelprovide[import, onchar=ids fonts]{brazilian}
\babelprovide[import, onchar=ids fonts]{english}

% Definição da Fonte Padrão (Noto Sans é mandatório para compilação segura)
\babelfont{rm}{Noto Sans}

% Pacotes Adicionais
\usepackage{enumitem}
\setlist[itemize]{label=-}
\usepackage{booktabs}
\usepackage{graphicx}
\usepackage{float}
\usepackage{listings}
\usepackage{xcolor}
\usepackage{titlesec}
\usepackage{microtype}
\usepackage{hyperref}

% Cores para Códigos
\definecolor{codegray}{rgb}{0.5,0.5,0.5}
\definecolor{codegreen}{rgb}{0,0.6,0}
\definecolor{codeblue}{rgb}{0,0,0.6}
\definecolor{backcolour}{rgb}{0.96,0.96,0.96}

% Definição da Linguagem YAML para Listings
\lstdefinelanguage{yaml}{
  keywords={true,false,null,y,n},
  keywordstyle=\color{codeblue}\bfseries,
  basicstyle=\ttfamily\footnotesize,
  sensitive=false,
  comment=[l]{\#},
  commentstyle=\color{codegreen}\ttfamily,
  stringstyle=\color{magenta}\ttfamily,
  moredelim=[l][\color{orange}]{\&},
  moredelim=[l][\color{magenta}]{*},
  moredelim=**[il][\color{red}]{:},
  morestring=[b]',
  morestring=[b]"
}

% Estilo para Listings (Código)
\lstdefinestyle{mystyle}{
    backgroundcolor=\color{backcolour},
    commentstyle=\color{codegreen},
    keywordstyle=\color{codeblue},
    numberstyle=\tiny\color{codegray},
    stringstyle=\color{magenta},
    basicstyle=\ttfamily\footnotesize,
    breakatwhitespace=false,
    breaklines=true,
    captionpos=b,
    keepspaces=true,
    numbers=left,
    numbersep=5pt,
    showspaces=false,
    showstringspaces=false,
    showtabs=false,
    tabsize=2,
    frame=single
}
\lstset{style=mystyle}

% Configuração de Metadados
\hypersetup{
    colorlinks=true,
    linkcolor=black,
    filecolor=magenta,
    urlcolor=blue,
    pdftitle={Implementação e Configuração de Servidor Web},
    pdfauthor={Fernanda Martins da Silva, Gabriel Maia Miguel, Samuel Oliveira Lopes},
}

% --- INÍCIO DO DOCUMENTO ---
\begin{document}

% --- CAPA ---
\begin{titlepage}
	\centering
	{\Large \textbf{Instituto Federal de Educação, Ciência e Tecnologia de São Paulo}}\\[0.5cm]
	{\large Campus São João da Boa Vista}\\[3cm]

	{\huge \textbf{Implementação e Configuração de um Servidor Web}}\\[0.5cm]
	{\Large Abordagem Prática com Docker Compose, NGINX e Aplicações Open Source}\\[3cm]

	\textbf{Autores:}\\[0.5cm]
	Fernanda Martins da Silva\\
	Gabriel Maia Miguel\\
	Samuel Oliveira Lopes\\[4cm]

	{\large \textbf{Relatório Técnico}}\\[0.5cm]
	Unidade Curricular: Redes de Computadores\\[3cm]

	\vfill
	Novembro de 2025
\end{titlepage}

% --- RESUMO ---
\thispagestyle{plain}
\begin{center}
	\Large \textbf{Resumo}
\end{center}
\vspace{0.5cm}

Este relatório detalha o processo de implementação e configuração de uma infraestrutura de servidores web baseada em microsserviços, utilizando tecnologias de código aberto (\textit{open source}) orquestradas via Docker. O projeto visa demonstrar a criação de um ambiente robusto e escalável, integrando Sistemas Gerenciadores de Banco de Dados (SGBD) heterogêneos (MariaDB 10.6 e PostgreSQL 15), um servidor \textit{gateway} e proxy reverso (NGINX), e aplicações web com pilhas tecnológicas distintas (Moodle 4.5 em PHP e Spring Petclinic em Java 25). A metodologia adotada privilegia a conteinerização com Docker Compose para assegurar a modularidade, o isolamento de redes e a portabilidade do ambiente. São apresentados os arquivos de configuração, as estratégias de otimização de imagens (\textit{multi-stage builds}) e a resolução de desafios relacionados ao roteamento de tráfego e persistência de dados.

\vspace{0.5cm}
\noindent \textbf{Palavras-chave:} Servidor Web, NGINX, Docker Compose, PostgreSQL, Moodle, Spring Boot, Microsserviços.

\newpage
\tableofcontents
\newpage

% --- CONTEÚDO ---

\section{Introdução e Objetivos}
\label{sec:objetivos}

A ubiquidade das aplicações web modernas exige infraestruturas que sejam não apenas funcionais, mas também seguras, modulares e de fácil manutenção. Este projeto insere-se no contexto prático da disciplina de Laboratório de Redes de Computadores, visando consolidar conhecimentos sobre protocolos HTTP, gestão de bases de dados e orquestração de serviços.

\subsection{Objetivo Geral}
O objetivo primordial é demonstrar a implementação técnica de um servidor web capaz de hospedar múltiplas aplicações concorrentes, com diferentes requisitos de execução (\textit{runtime}), em um único ambiente coeso e acessível via rede local (LAN).

\subsection{Objetivos Específicos}
Para atingir o objetivo geral, a equipe definiu os seguintes passos:
\begin{enumerate}
	\item \textbf{Gestão de Dados:} Implementação e configuração de SGBDs relacionais distintos (MariaDB e PostgreSQL) em contêineres isolados.
	\item \textbf{Aplicações Heterogêneas:}
	      \begin{itemize}
		      \item Hospedagem de uma aplicação Java Web (Spring Petclinic) utilizando as versões mais recentes do Java (v25).
		      \item Configuração de uma plataforma LMS (\textit{Learning Management System}) baseada em PHP (Moodle 4.5).
	      \end{itemize}
	\item \textbf{Roteamento de Tráfego:} Configuração de um servidor NGINX atuando como Proxy Reverso para gerenciar as requisições na porta 8080 e distribuí-las pelos serviços internos.
	\item \textbf{Documentação:} Elaboração técnica dos artefatos de configuração (\texttt{Dockerfile}, \texttt{compose.yml}, \texttt{nginx.conf}).
\end{enumerate}

\section{Fundamentação Teórica e Arquitetura}
A solução foi desenhada seguindo o padrão de arquitetura de microsserviços, onde cada função do sistema é encapsulada na sua própria unidade de execução (contêiner).

\subsection{Tecnologias Adotadas}
A escolha das tecnologias baseou-se na estabilidade, suporte da comunidade e conformidade com padrões \textit{open source}. A Tabela \ref{tab:versoes} resume a pilha tecnológica extraída dos arquivos de configuração.

\begin{table}[ht]
	\centering
	\caption{Versões e Tecnologias Utilizadas}
	\label{tab:versoes}
	\vspace{0.2cm}
	\begin{tabular}{lp{6cm}l}
		\toprule
		\textbf{Serviço}     & \textbf{Tecnologia/Imagem}           & \textbf{Papel no Sistema}     \\
		\midrule
		\textit{Web Gateway} & \texttt{nginx:latest}                & Proxy Reverso e Estáticos     \\
		LMS App              & \texttt{brandkern/moodle:4.5-latest} & Plataforma de Ensino          \\
		LMS DB               & \texttt{mariadb:10.6}                & Persistência para o Moodle    \\
		Java App             & \texttt{eclipse-temurin:25} (Custom) & Aplicação Spring Petclinic    \\
		Java DB              & \texttt{postgres:15-alpine}          & Persistência para o Petclinic \\
		\bottomrule
	\end{tabular}
\end{table}

\subsection{Topologia de Rede e Isolamento}

Um aspecto crítico da segurança em Docker é o isolamento de redes. Em vez de utilizar a rede padrão (\textit{default bridge}), foram criadas duas redes internas distintas para garantir que as bases de dados sejam acessíveis apenas pelas suas respectivas aplicações:

\begin{itemize}
	\item \textbf{\texttt{backend}}: Rede dedicada à pilha Java. Conecta exclusivamente o NGINX, a aplicação Petclinic e a base de dados PostgreSQL.
	\item \textbf{\texttt{moodle\_net}}: Rede dedicada à pilha PHP. Conecta exclusivamente o NGINX, o Moodle e a base de dados MariaDB.
\end{itemize}

O contêiner \texttt{web} (NGINX) é o único elemento da arquitetura conectado a ambas as redes, atuando como uma ponte de aplicação (\textit{Application Layer Gateway}) segura entre o mundo externo e os serviços internos.

\section{Implementação Detalhada}

\subsection{Roteamento (NGINX)}
O NGINX foi configurado para ser o único ponto de contato com o mundo exterior, escutando na porta \textbf{8080}. A sua função de Proxy Reverso é vital para permitir que múltiplas aplicações compartilhem o mesmo endereço IP e porta. Além disso, foi configurado um volume específico para servir a documentação estática do projeto:
\begin{lstlisting}[language=bash]
./web/public:/usr/share/nginx/html
\end{lstlisting}
Esta diretiva mapeia os arquivos locais da equipe para o diretório padrão de publicação do servidor web.

\section{Análise Detalhada dos Artefatos de Configuração}
Nesta seção, a equipe apresenta a análise técnica dos arquivos de configuração desenvolvidos, explicando a função de cada diretiva.

\subsection{Orquestração: \texttt{compose.yml}}
Este arquivo define a infraestrutura como código (IaC).
\begin{itemize}
	\item \textbf{Serviços (\texttt{services}):}
	      \begin{itemize}
		      \item \texttt{web}: Utiliza a imagem \texttt{nginx:latest}. A porta 8080 do host é mapeada para a 80 do contêiner. A dependência \texttt{depends\_on} garante que o proxy inicie apenas após as aplicações estarem prontas.
		      \item \texttt{petclinic}: Construído a partir de um \texttt{Dockerfile} local (contexto \texttt{./spring-petclinic}). Utiliza a variável de ambiente \texttt{SPRING\_PROFILES\_ACTIVE=postgres} definida no arquivo externo \texttt{config.env} para ativar o driver JDBC correto.
	      \end{itemize}
	\item \textbf{Volumes (\texttt{volumes}):}
	      \begin{itemize}
		      \item Volumes nomeados como \texttt{moodle\_db\_data} e \texttt{petclinic\_db\_data} garantem a persistência dos dados. O Docker gerencia o local de armazenamento no host, protegendo os dados contra a exclusão acidental dos contêineres.
	      \end{itemize}
\end{itemize}

\subsection{Construção da Aplicação Java: \texttt{Dockerfile}}
O arquivo de construção do \texttt{petclinic} é um exemplo clássico e eficiente do padrão \textit{Multi-stage Build}. Esta técnica divide o processo de criação da imagem em fases distintas para otimizar o tamanho e a segurança do artefato final.

\begin{lstlisting}[language=bash, caption=Dockerfile com Multi-stage Build]
# Estágio 1: Build
FROM eclipse-temurin:25-jdk AS build
WORKDIR /workspace
COPY . .
RUN ./mvnw package -DskipTests

# Estágio 2: Runtime
FROM eclipse-temurin:25-jre
WORKDIR /app
COPY --from=build /workspace/target/*.jar app.jar
ENTRYPOINT ["sh", "-c", "java $JAVA_OPTS -Dserver.servlet.context-path=/petclinic -jar /app/app.jar"]
\end{lstlisting}

\textbf{Explicação Técnica:}
\begin{enumerate}
	\item \textbf{Estágio de Compilação (\texttt{AS build}):} Inicia-se com uma imagem base completa (JDK 25), que contém compiladores e ferramentas de desenvolvimento. O código fonte é copiado e o Maven Wrapper (\texttt{mvnw}) é executado para compilar a aplicação e gerar o arquivo \texttt{.jar}. Note que este estágio é pesado, contendo código fonte e dependências de build.

	\item \textbf{Estágio de Execução (\texttt{Runtime}):} Inicia-se um novo estágio com uma imagem base minimalista (JRE 25), que contém apenas o ambiente de execução Java, sem compiladores.

	\item \textbf{Transferência de Artefato (\texttt{COPY --from=build}):} Apenas o arquivo compilado (\texttt{.jar}) é copiado do estágio anterior para a nova imagem. Todo o código fonte, cache do Maven e ferramentas de desenvolvimento são descartados.

	\item \textbf{Resultado:} Obtém-se uma imagem final significativamente menor e mais segura para produção, pois reduz a superfície de ataque ao não conter ferramentas desnecessárias.
\end{enumerate}

\subsection{Configuração do Proxy: \texttt{nginx.conf}}
O arquivo define o comportamento do servidor de borda, com estratégias distintas para cada aplicação:

\begin{itemize}
	\item \textbf{Rota Petclinic:} A diretiva \texttt{proxy\_pass http://petclinic:8080;} encaminha o tráfego para o serviço Java. O uso do nome do serviço (\texttt{petclinic}) é resolvido pelo DNS interno do Docker.

	\item \textbf{Rota Moodle:} A diretiva \texttt{proxy\_pass http://moodle;} foi configurada \textbf{sem a barra final} (\textit{trailing slash}).
	      \begin{itemize}
		      \item \textit{Justificativa Técnica:} Conforme observado nos arquivos de configuração, omitir a barra instrui o NGINX a anexar a URI original da requisição ao encaminhá-la para o upstream. Isso garante que requisições como \texttt{/moodle/login/index.php} sejam recebidas corretamente pelo container Apache interno.
	      \end{itemize}

	\item \textbf{Headers de Transparência:}
	      \begin{itemize}
		      \item \texttt{proxy\_set\_header Host \$http\_host}: Preserva o cabeçalho original, essencial para que a aplicação de backend saiba qual domínio o cliente acessou.
		      \item \texttt{X-Forwarded-For}: Adiciona o IP real do cliente à requisição, permitindo logs de auditoria corretos e controle de acesso baseado em IP na aplicação final.
	      \end{itemize}
\end{itemize}

\subsection{Variáveis de Ambiente: \texttt{config.env}}
Aderindo aos princípios da metodologia \textit{12-Factor App}, as configurações sensíveis foram externalizadas.
\begin{itemize}
	\item \texttt{POSTGRES\_URL}: Define a string de conexão JDBC, apontando para o hostname \texttt{petclinic-db}.
	\item \texttt{WWWROOT}: Configuração crítica para o Moodle, informando-o de que ele está sendo acessado via \texttt{http://localhost:8080/moodle}, permitindo a geração correta de URLs absolutas para recursos estáticos (CSS/JS).
\end{itemize}

\section{Desafios e Soluções}
Durante a implementação, a equipe deparou-se com desafios técnicos, especificamente relacionados ao \textit{Context Path}.

\textbf{Problema:} As aplicações assumem nativamente a execução na raiz (\texttt{/}). Ao servi-las em subdiretórios (\texttt{/moodle} e \texttt{/petclinic}), links internos quebravam.

\textbf{Solução:} Reconfiguração explícita:
\begin{itemize}
	\item \textbf{Java:} Adição da propriedade \texttt{-Dserver.servlet.context-path=/petclinic} no \texttt{ENTRYPOINT} do Dockerfile.
	\item \textbf{Moodle:} Definição correta da variável \texttt{WWWROOT} no arquivo de ambiente, alinhada com a configuração do bloco \texttt{location} do NGINX.
\end{itemize}

\section{Conclusão e Trabalhos Futuros}
A infraestrutura implementada cumpriu integralmente os requisitos propostos. A orquestração via Docker Compose garantiu um ambiente reprodutível e modular. A segmentação de redes demonstrou-se eficaz para a segurança, e a estratégia de \textit{multi-stage build} no Dockerfile da aplicação Petclinic assegurou uma entrega de software otimizada.

Como trabalhos futuros, sugere-se a implementação de certificados SSL/TLS (HTTPS) para garantir a confidencialidade dos dados em trânsito.

\vspace{0.5cm}
\noindent\textbf{Repositório do Projeto:} \href{https://github.com/sunref/redc}{https://github.com/sunref/redc}

\end{document}
