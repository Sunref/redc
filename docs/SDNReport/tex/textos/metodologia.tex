\chapter{Pesquisa}
\label{cap:pesquisa}

\section{O que é a SDN?}
Nos últimos anos com o crescimento de tablets, smartphones e outros dispositivos de transmissão multimídia
surgiu a necessidade de controle e operação de rede, essencial para suprir as demandas desses sistemas.
Portanto, nada mais é que uma arquitetura de redes entre computadores, visando gerenciar serviços de rede
utilizando de softwares em vez de dispositivos especializados para esse tipo de controle.

Sendo um sistema centralizado, é capaz de reservar ou preparar recursos para que a aplicação não tenha
obstáculos técnicos de hardware, possuíndo monitoramento inteligente que é feito para ser adaptável automaticamente
de acordo com o estado da rede, digitalizando a mesma.

A fim e lidar com melhores aplicações em nuvem, é capaz de automatizar, escalar e otimizar redes redes publicas e privadas
de serviços, além de banco de dados. Escalável com mudanças contínuas pelas quais serviços de operadoras e provedores de
internet não conseguem acompanhar \cite{stefanini_performance2025}.
Alguns exemplos de SDN incluem a OpenDaylight\footnote{Disponível em: https://www.opendaylight.org/},
ONUS\footnote{Disponível em: https://opennetworking.org/onos/}, Ryu\footnote{Disponível em: https://ryu-sdn.org/},
VMware NSX\footnote{Disponível em: https://www.vmware.com/}

\section{Como Funciona}
A SDN é formada por componentes que podem ou não estar estarem localizadas na mesma área física.
A fim de eliminar funções de de roteamento e encaminhamento de pacotes, a SDN implementa controladores
e os coloca acima de hardwares de rede na nuvem ou localmente, permitindo gerenciamento de rede
diretamente \cite{ibm_sdn}.

Os componentes que compõem uma SDN consistem em:
\begin{itemize}
	\item \textbf{Aplicações:} São as encarregadas por transmitir informações ou solicitações de disponibilidadealocação
	de rede. Sendo composta por dois tipos de interface de programação de aplicações (API), é notável citar
	\textit{Southbound} e \textit{Northbound}. Os controladores podem programar e configurar dispositivos de rede nessa
	API (\textit{Southbound}), recuperando informações sobre estados e topologia, recebendo notificações sobre congestionamento de pacotes
	e falhas de link. Já a \textit{Northbound} executa as mesmas funções que a API anterior, com diferença em viabilizar automatização de tarefas
	de gerenciamento de redes, facilitar a integração de sistemas em nuvem e outros tipos de aplicações.
	\item \textbf{Controladores:} Responsável por implementar funções de controle de redes e coordenar a comunicação com aplicações determinando
	o tráfego de pacotes de dados, os controladores oferecem uma perspectiva mais centralizada da rede, armazenando informações sobre o estado da
	mesma e toma decisões de como gerenciar dispositivos de rede conforme suas politicas.
	\item \textbf{Dispositivos de Rede:} São \textit{switches}, roteadores e pontos de acesso que fazem o fluxo de pacotes e recebem as instruções dos controladores e podem oferecer suporte a funcionalidades, como encaminhamento baseado em fluxo, qualidade de serviço e engenharia de tráfego. Nas SDN esses dispositivos podem ser simplificados e padronizados.
	\item \textbf{Sistema MANO\footnote{Management and Orchestration}:} MANO, ou gerenciamento de orquestração, interage com o controlador de SDN por meio da API \textit{Northbound}
	automatizando a utilização de recursos para rede e garantindo o autodesempenho e disponibilidade de serviços.
\end{itemize}
\cite{redhat_sdn}

\section{Tipos}
Existem quatro tipos de SDN que são considerados os principais. São eles:

\begin{itemize}
	\item \textbf{SDN aberta:} Os protocolos públicos são usados como controle de dispositivos tanto físicos quanto virtuais, e são
	responsáveis pelo roteamento dos pacotes de dados.
	\item \textbf{SDN de API:} Nesses casos, geralmente a \textit{Southbound} fica responsável pela organização e controle do fluxo para
	cada dispositivo.
	\item \textbf{Modelo de sobreposição:} Uma rede virtual acima do hardware físico oferecendo túneis que estabelecem canais de comunicação
	com centro de processamento de dados (CPD).
	\item \textbf{Modelo híbrido:} Combina as SDNs com redes tradicionais, atribuindo o protocolo certo para cada trafego. Frequentemente usada como complemento as SDNs originais.
\end{itemize}
\cite{ibm_sdn}

\section{Vantagens e Desvantagens}
As SDNs centralizam o controle e gerenciamento de rede, isso oferece vantagens que outras abordagens de rede não possuem.
Podemos citar:

\subsection{Agilidade e Flexibilidade}
Permite o balanceamento de fluxo de tráfego de acordo com a necessidade e do uso, reduzindo latência, aumentando a eficiência da rede.
As operadoras de rede também tem mais controle sobre a mesma, podendo alterar suas configurações, garantir recursos e expandir sua capacidade
\cite{ibm_sdn}.

\subsection{Redução de Custos}
Como as SDNs mantém sempre um tráfego contínuo, mesmo sendo um alto investimento a se fazer, gera ao departamento de TI (Tecnologia
da Informação) um alívio, reduzindo custos e melhorando a eficiência de serviços ao consumidor final \cite{stefanini_performance2025}.

\subsection{Controle}
Permite aos administradores que definam suas políticas a partir de um local central para controlar na rede os acessos e suas medidas
de seguranças. Sendo aplicável em nuvem publica, híbrida, privada e multinuvem \cite{ibm_sdn, stefanini_performance2025}.

\subsection{Simplicidade}
Podendo se basear em um único protocolo de comunicação com uma ambla variedade de dispositivos de hardware, oferecendo flexibilidade na escolha de
dispositivos de rede, gerando simplicidade \cite{ibm_sdn}.

\subsection{Modernização de Telecomuniçaões}
Combinado à maquinas virtuais e virtualização de redes, permite que as operadoras forneçam separação de rede e controle distinto aos clientes.
Auxilia os operadores a melhorar sua escalabilidade e fornecer largura de banda sob demanda ao clientes\cite{ibm_sdn}.

Porém, ainda é sucetível a erros e problemas, sendo o mais comum a ser citado o:

\subsection{Risco de Centralização}
Por ser um sistema centralizado, há um único potencial ponto de falha vulnerável, pois, como as SDNs induzem a criação de novos pontos de rede,
a mesma fica sucetível a vulnerabilidades e ataques cibernéticos. Algumas SDNs também são de código aberto, o que facilita a implementação de
código malicioso \cite{ibm_sdn, ufrj_sdn_2018}.